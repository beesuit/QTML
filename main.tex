\documentclass{article}
\usepackage[ruled,linesnumbered,lined]{algorithm2e}
\usepackage[braket]{qcircuit}
\usepackage[final]{listings}
\usepackage{caption,subcaption}
\usepackage{graphicx}
\usepackage[margin=3cm]{geometry}
\usepackage{amsmath}
\usepackage[utf8]{inputenc}

\lstset{numbers=left, numberstyle=\tiny, stepnumber=2, numbersep=5pt,
basicstyle=\small, stringstyle=\ttfamily, showstringspaces=false}

\title{}
\author{}

\begin{document}
\maketitle
\begin{abstract}
  A probabilistic quantum memory is the quantum equivalent of a content-addressable associative memory. It is capable of storing and recovering large amounts of information using quantum superposition. The retrieval algorithm is probabilistic, the output is obtained through the hamming distance between the input and the patterns stored on the memory. Probabilistic quantum memories can be used as quantum classifiers in quantum machine learning algorithms. The main limitation of this probabilistic quantum memory is the necessity to measure the memory in order to recover information; the measurement breaks the quantum system and the information must be reinserted on the memory. However, some applications with just a single shot of the recovering algorithm and the application of the quantum probabilistic memory can lead to super-polynomial advances in heuristic search when a maximum value is known. 
  
  
  In this work we propose a modified version of the probabilistic quantum memory in which a scale parameter $\phi$ is given to the retrieval algorithm. The scale parameter is used when the hamming distance between the input pattern and the patters stored on the memory is computed. We create a classical/quantum protocol to implemented the memory on a real quantum device using a five bit quantum computer. Through classical experimentation we evaluated the influence of the scale parameter on the pattern classification capabilities of the memory. The proposed parametric quantum probabilistic memory allow the adjustment of the quantum memory to a given task. Hence, it can better suit the task of machine learning model evaluation, once we can adjust the scale parameter to improve the classification accuracy.
\end{abstract}

\bibliographystyle{unsrt}
\bibliography{bibliografia}

\end{document}
